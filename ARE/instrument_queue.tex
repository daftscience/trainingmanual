When Cerner gets results from the instruments that cannot be \textsc{Auto-verified} it will send them to the \textsc{Instrument Queue}.

The \textsc{Instrument Queue} mode is used to review questionable results before \textsc{Verifying} them.

This mode will be useful in departments where many results are sent from the analyzers.\sidenote{Chemistry, Coagulation, and possibly Urinalysis.}\\

\prettyimage{width=\textwidth}{graphics/instrument_queue}

\section{Opening Instrument Queue}

\paragraph{Open} \gls{are}.\marginnote[-2\baselineskip]{\prettyimage{width=\linewidth}{graphics/instrument_queue_menu}}

\paragraph{Click} \boldcap{Mode} from the menu bar.

\paragraph{Select} \boldcap{Instrument Queue Mode}.

\section{Cosmetic Differences}

The difference between \textsc{Instrument Queue} and \textsc{Accession Mode} is the selection box.\\

\prettyimage{width=.6\textwidth}{graphics/instrument_queue_changes}

\begin{description}
    \bolditem{Test Site} The Instrument Queue you'd like to view.
    \bolditem{Procedure} This field is used to limit the results to a specific test.\sidenote{This is an optional field.}
    \bolditem{Accession} This field shows the \gls{accn} you're viewing.\sidenote{If it's available.}
\end{description}

\section{Using Instrument Queue}

\paragraph{Enter} the \instrument, \bench, \sect or \subsect.\sidenote{For more information: \checkrouting}

\paragraph{Click} \btn{graphics/start}\\

\prettyimage{width=.8\textwidth}{graphics/instrument_queue_started}

\newthought{The queue will} remain here until results are sent over which needs to be viewed.

\prettyimage{width=\textwidth}{graphics/instrument_queue_results}\marginnote{\note{This Screenshot was taken before Autoverify rules were updated.}}

\paragraph{Review} the results.

\paragraph{Enter} any appropriate comments.

\paragraph{Click} \btn{graphics/verify}.

\subsection{Moving To the Next Result}

If you're not ready to \textsc{Verify} the result you can move to the next item in the queue.

\paragraph{Click} the \btn{graphics/down} button on the tool bar.


\subsection{Go To the Start of the Queue}

When you reach the end of the list, the queue will be blank.

\paragraph{Click} \btn{graphics/stop} then,

\paragraph{Click} \btn{graphics/start}

\newthought{This will bring} you back to the start of the queue.


\subsection{Canceling Orders}

If the results are questionable, we can cancel the orders using \gls{orv} or \gls{pi}.

