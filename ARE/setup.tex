\section{Setup Result View}

The \textsc{Results Spreadsheet} of \Gls{are} can be modified to display additional information. Specifically, the \textsc{Reference Range} column\sidenote{It's really handy.} should be added.

\paragraph{Open} \gls{are} by clicking the \appicon{are} icon from the \gls{ab}.

\paragraph{Click} \boldcap{View} from the menu bar.\\

\prettyimage{width=.5\linewidth}{graphics/view_menu2}

\paragraph{Click} \boldcap{Customize\ldots}

\paragraph{Click} \boldcap{Result Display\ldots}\\

\prettyimage{width=.5\linewidth}{graphics/menu_results_display}

\paragraph{Check \faCheckSquareO} \textsc{Reference Range}.

\paragraph{Check \faCheckSquareO} \textsc{Service Resource Display}.\sidenote{This option will add a column to \gls{are}. When instruments perform results, the Instrument ID will be present in this column.}\\

\prettyimage{width=.5\textwidth}{graphics/result_display}

\paragraph{Un-Check \faSquareO} the other options.\sidenote{Or leave them Un-checked. They really don't add much.}\\

\newthought{Now, next to the} sample status is a column which lists the reference ranges for each assay.\\

\prettyimage{width=\textwidth}{graphics/result_sheet}


\section{Assay Sorting}
By default, the assays are sorted by \textsc{Orderable}. This sorting method is very inconvenient if the assays are performed in multiple departments.

Sorting by \textsc{Service Resource} will group assays based on where they're performed.

\paragraph{Click} \boldcap{View} from the menu bar.\\

\prettyimage{width=.5\linewidth}{graphics/view_menu2}

\paragraph{Click} \boldcap{Customize\ldots}

\paragraph{Click} \boldcap{Sort assays}.

\paragraph{Click} \boldcap{By Service Resource}.\\

\prettyimage{width=\textwidth}{graphics/service_resource}

\newthought{Now, assays are will} be sorted by \textsc{Bench}, or testing location, making \gls{are} much easier to use.


