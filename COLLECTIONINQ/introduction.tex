\Gls{colinq} is a Cerner application used primarily for troubleshooting orders which have not reached the laboratory.

Sites with phlebotomy can also use this application to look ahead at future collections.

\newthought{Open \gls{colinq}} by clicking the \appicon{collections_inquiry} icon from the \gls{ab}.\sidenote{\checkref{ch:ab_addapp}{\refptch{part:ab}{ch:ab_addapp} }{Refer to the \textsc{App-Bar Procedure} } if you need help adding it.}\\

\prettyimage{width=\textwidth}{graphics/col_inq}

\newthought{There are five} tabs in the \gls{colinq} window. Each is used to get different information.

\begin{table}
    \scshape
    \begin{tabular}{ll}
        \boldcap{Tab} & \boldcap{Useful Application} \\
        \hline
         Location &  Easy way to re-print morning collection labels.\\
         List &  Helpful to track inbound samples from other locations.\\
         Accession &  Usefulness hasn't been determined.\\
         Patient &  Helpful when nurses are having label difficulties.\\
         Missed &  Usefulness hasn't been determined.\\
        \hline
    \end{tabular}
    \caption{Collection Inquiry Modes}
    \label{table:col_inq_modes}
\end{table}
