
There are certain situations in which orders need to be added on to an existing accession number (\eg{Glomerular Filtration Rate.}) This is done in \gls{doe} through the \textsc{\gls{accnaddon}} screen.

\newthought{Switch to } \textsc{\gls{accnaddon}} mode:

\paragraph{Click} the \includegraphics[height=1em]{graphics/accession_addon_icon.png} icon from the toolbar.\\

\prettyimage{width=\textwidth}{graphics/toolbar_accn_addon.png}

\newthought{\gls{doe} will update.}

The main difference between this mode, and the normal \textsc{Order Entry} mode is that the \textsc{Patient Identifier} field gets replaced with an accession number field.\\

\noindent
\begin{tikzpicture}
\begin{scope}%[xshift=1.5cm]
    \node[anchor=south west,inner sep=0] (image) at (0,0) {\prettyimage{width=\textwidth}{graphics/doe_accession_addon.png}};
    \begin{scope}[x={(image.south east)},y={(image.north west)}]
        \draw [-stealth, line width=3pt, deeppurple400] (0.35,1.05) to[out=-90, in=0] (0.12,0.76);
    \end{scope}
\end{scope}
\end{tikzpicture}

\section{Adding on Orders}

\paragraph{Enter} the \textsc{\glsname{accn}}\sidenote{The accession number of the sample the orders will be added to.} into the \gls{accn} field.\accntip\\

% \vspace{1.5em}

\noindent
\begin{tikzpicture}
\begin{scope}%[xshift=1.5cm]
    \node[anchor=south west,inner sep=0] (image) at (0,0) {\prettyimage{width=\textwidth, trim={0, 330pt, 0, 0}, clip}{graphics/doe_accession_addon.png} };
    \begin{scope}[x={(image.south east)},y={(image.north west)}]
        \draw [-stealth, line width=3pt, deeppurple400] (0.19,1.05) to[out=-92, in=0] (0.12,0.17);
    \end{scope}
\end{scope}
\end{tikzpicture}

\paragraph{Press} {\faKeyboardO} \textsc{Enter} on the keyboard.

\paragraph{Enter } the test to be ordered in the \textsc{Orderable Field}.\marginnote[2\baselineskip]{We're looking for a Basic Metabolic Panel.}\\

\noindent
\begin{tikzpicture}
\begin{scope}%[xshift=1.5cm]
    \node[anchor=south west,inner sep=0] (image) at (0,0) {\prettyimage{width=\textwidth, trim={0, 300pt, 0, 0}, clip}{graphics/accn_retrieved.png}};
    \begin{scope}[x={(image.south east)},y={(image.north west)}]
        \draw [-stealth, line width=3pt, deeppurple400] (0.15,1.05) to[out=-92, in=0] (0.12,0.15);
    \end{scope}
\end{scope}
\end{tikzpicture}


\paragraph{Search} for the new test.
%-------------------------------------------------------------------------------------------------------------------------------------
% New Page!
% ------------------------------------------------------------------------------------------------------------------------------------
\section{To Override or Not to Override}
When orders are added onto an existing \gls{accn} they need to be assigned to a container.\sidenote{\checkref{sec:accn_cid}{\refptch{part:accn}{sec:accn_cid}}{Refer to the \textsc{Accession Numbers} Procedure.}} This section will help walk you through the processes. It may sound more confusing than it is.\sidenote{Sorry \faFrownO}

Before you begin, it's helpful to know which containers are on the original \gls{accn}. To find out, we can use our buddy \textsc{\gls{ci}}.\sidenote{\checkref{part:ci}{\refptapp{part:ci}{container_inquiry}.}{ Refer to the \boldcap{\gls{ci}} procedure for more information.}}\\


\newthought{If you're asked} to ``\textsc{Override},'' you can think of the question like this:
\begin{quote}
\itshape
``Can the add-on use an existing container?''\sidenote{``Can a Basic Panel be run on a Lavender?''\\\textit{or}\\``Can a Basic Panel be run on a Gold Top?''}
\end{quote}
\prettyimage{width=.5\textwidth}{graphics/no_valid_container.png}

\clearpage
\section{Not Overriding}\marginnote{You clicked ``No'' when asked to override.}
\paragraph{Select} the appropriate container.\\
\begin{tikzpicture}
\begin{scope}%[xshift=1.5cm]
    \node[anchor=south west,inner sep=0] (image) at (0,0) {\prettyimage{width=.8\textwidth}{graphics/new_container_2.png}};
    \begin{scope}[x={(image.south east)},y={(image.north west)}]
        \draw [-stealth, line width=3pt, deeppurple400] (0.15,1.05) to[out=-92, in=90] (0.07,0.5);
    \end{scope}
\end{scope}
\end{tikzpicture}\\
\paragraph{Click} \btn{graphics/ok_button} and continue the order as normal.\\

\section{Overriding}\marginnote{You clicked ``Yes'' when asked to override.}

\paragraph{Choose} an existing container.\marginnote[2\baselineskip]{\important{Only \faCheckSquareO one of these boxes. Selecting multiple containers will assign the order to multiple containers, and that's just asking for trouble.}}\\
\vspace{.5em}
\begin{tikzpicture}
\begin{scope}%[xshift=1.5cm]
    \node[anchor=south west,inner sep=0] (image) at (0,0) {\prettyimage{width=.8\textwidth}{graphics/override_window.png}};
    \begin{scope}[x={(image.south east)},y={(image.north west)}]
        \draw [-stealth, line width=3pt, deeppurple400] (0.15,1.05) to[out=-92, in=90] (0.07,0.76);
    \end{scope}
\end{scope}
\end{tikzpicture}\\
\vspace{.5em}
\centerline{\textit{\textbf{--OR--}}}

\paragraph{Click} \boldcap{New Container}.\marginnote{This is the same as clicking ``No''  to the override question.}\marginnote[3\baselineskip]{\important{Do not \faCheckSquareO \textit{both} ``New Container\ldots'' \textit{AND} an existing container. Again, just asking for trouble.}}\\
\vspace{.5em}
\begin{tikzpicture}
\begin{scope}%[xshift=1.5cm]
    \node[anchor=south west,inner sep=0] (image) at (0,0) {\prettyimage{width=.8\textwidth}{graphics/override_window.png}};
    \begin{scope}[x={(image.south east)},y={(image.north west)}]
        \draw [-stealth, line width=3pt, teal400](0.15,1.05) to[out=-92, in=90] (0.04,0.25);
    \end{scope}
\end{scope}
\end{tikzpicture}
\vspace{.5em}

\paragraph{Click} \btn{graphics/ok_button} to add the test onto the selected container.

\section{Submitting Orders}

Finally, when all the appropriate orders have been added:

\paragraph{Click} the \textsc{Submit Orders} icon \includegraphics[height=1em]{graphics/submit_order_icon.png} from the tool bar. \sidenote{\hotkey{\textbf{Ctrl+O} will also submit orders.}}\\

\section{Return to Order Entry Mode}

To go back to the normal \textsc{Order Entry Mode}:

\paragraph{Click} the \includegraphics[height=1em]{graphics/order_entry_icon.png} icon from the toolbar.\\

\prettyimage{width=\textwidth}{graphics/order_entry}

Alternatively, closing and re-opening \gls{doe} will also work.

\dupnote