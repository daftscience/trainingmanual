%-------------------------------------------------------------------------------------------------------------------------------------
% New Page!
% ------------------------------------------------------------------------------------------------------------------------------------
From \acrlong{doe} there are two ways to search for a patient.\\
The quickest way to find a patient is using a \textsc{Unique Patient Identifier} such as a \gls{mrn}, or \gls{fin}.

If that information is unavailable the \textsc{Encounter Search} window can be used to find patients using information such as Partial name, date of birth \textit{etc\ldots}
\vfill
\section{Using a Unique Patient Identifier}\label{sec:doe_pid}
\vfill
\paragraph{Enter} the \boldcap{Patient Identifier} in the \textsc{Patient Identifier} field.\sidenote{Make sure that the \textsc{Patient Identifier} entered matches the description above the text box.}\\

\noindent
\begin{tikzpicture}
\begin{scope}%[xshift=1.5cm]
    % \node [anchor=west] (pif) at (0.2,0.3);
    \node[anchor=south west,inner sep=0] (image) at (0,0) {\prettyimage{width=.8\textwidth}{graphics/doe_patient_search}};
    \begin{scope}[x={(image.south east)},y={(image.north west)}]
        \draw [-stealth, line width=3pt, deeppurple400] (0.08,1.05) to[out=-90, in=-185] (0.02,0.28);
    \end{scope}
\end{scope}
\end{tikzpicture}%
\paragraph{Click } \btn{graphics/elips_button.png} to pull up the patient.\sidenote{\hotkey{ Hitting \boldcap{Enter} also works}}

\vfill
\newthought{If the patient has} only one encounter, \gls{doe} will use that encounter. The demographics in \gls{doe} will update with the selected patient.

However, if there are multiple encounters, or \gls{doe} could not find the patient, then the ``Encounter Search'' window will appear.\\
% \sidenote{\textit: ``Using the Encounter Search Window''}

\clearpage
%-------------------------------------------------------------------------------------------------------------------------------------
% New Page!
% ------------------------------------------------------------------------------------------------------------------------------------
\section{Using the Encounter Search Window}
The \textsc{Encounter Search Window} can be used to find patients in the event that a \textsc{Unique Patient Identifier} is missing.\\

\begin{tikzpicture}
\node [anchor=west] (pif) at (1, 6.9) {\scshape{Search Fields}};
\node [anchor=west] (cf) at (3, 6.4) {\scshape{Persons List}};
\node [anchor=west] (el) at (5, 5.9) {\scshape{Encounters List}};
\begin{scope}[yshift=-.7cm]
    \node[anchor=south west,inner sep=0] (image) at (0,0) {\prettyimage{width=.9\textwidth}{graphics/encounter_results}};
    \begin{scope}[x={(image.south east)},y={(image.north west)}]
        \draw [-stealth, line width=3pt, deeporange400] (pif) to[out=270, in=-255] (0.1,0.9);
        \draw [-stealth, line width=3pt, teal400] (cf) to[out=270, in=90] (0.3,0.89);
        \draw [-stealth, line width=3pt, deeppurple400] (el) to[out=270, in=90] (0.3,0.5);
    \end{scope}
\end{scope}
\end{tikzpicture}%

\newthought{The Encounter Search Window} is divided into three sections:
\begin{description}%
    \bolditem{Search Fields} This is where the search criteria are entered.%
    \bolditem{Persons List} A list of persons matching the search criteria.%
    \bolditem{Encounters List} A list of encounters for the Person selected in the ``Persons List.''%
\end{description}


\newthought{To open the Encounter Search} window:

\paragraph{Click}  \btn{graphics/elips_button} to open the ``Encounter Search'' window.\sidenote{\hotkey{The \boldcap{Enter} will also work.}}

\begin{tikzpicture}
\begin{scope}%[xshift=1.5cm]
    \node[anchor=south west,inner sep=0] (image) at (0,0) {\prettyimage{width=.8\textwidth}{graphics/doe_patient_search}};
    \begin{scope}[x={(image.south east)},y={(image.north west)}]
        \draw [-stealth, line width=3pt, cyan500] (0.13,1.05) to[out=-90, in=-185] (0.38,0.28);
    \end{scope}
\end{scope}
\end{tikzpicture}%

\clearpage
\paragraph{Enter} any known information into the search fields.\sidenote{\important{Be as specific as possible. If the search is too vague Cerner will simply display ``Too many persons found. Please enter more filter data.''}}\marginnote[4\baselineskip]{\trouble{If any of the search fields are populated when the window opens, but the Persons List says ``No Persons Found'' \refpt{sec:doe_ptlookup_trouble}}}\\

\prettyimage{width=.75\textwidth}{graphics/encounter_search}

\paragraph{Click} \btn{graphics/search_button} to search.

\paragraph{Select} the appropriate patient from the  Persons List.\marginnote[4\baselineskip]{These are the patients which match the search criteria.}\marginnote[3\baselineskip]{These are the encounters for the, currently selected, patient.}\\

\prettyimage{width=.75\textwidth}{graphics/encounter_results}
\paragraph{Select} the appropriate encounter from the Encounters List.
\paragraph{Click} \btn{graphics/doe_ok_button}\marginnote[4\baselineskip]{\gls{doe} will update with the patient's demographics, and orders can now be placed.}\\

\prettyimage{width=.75\textwidth}{graphics/encounter_results.png}

% \clearpage
\section{Troubleshooting Tips\label{sec:doe_ptlookup_trouble}}

If you're having trouble finding a patient, don't fret. Here are some things to look for when the ``Encounter Search'' window says: ``No persons found'' but the patient \textit{is} in the system.

\subsection{Check the Search Fields}
\newthought{In the example below, } the ``Patient Identifier'' field was set to search for a patient name, however the user entered a \gls{fin}.\marginnote[3\baselineskip]{The ``Encounter Search'' window will say ``No persons found.''}\\

\begin{tikzpicture}
\begin{scope}%[xshift=1.5cm]
    \node[anchor=south west,inner sep=0] (image) at (0,0) {\prettyimage{width=.8\textwidth}{graphics/wrong_identifier_location.png}};
    \begin{scope}[x={(image.south east)},y={(image.north west)}]
        \draw [stealth-stealth, line width=3pt, red400] (0.03,0.75) to[out=60, in=90] (0.26,0.8);
    \end{scope}
\end{scope}
\end{tikzpicture}%

\begin{quote}
\newthought{\textbf{Solution:}} This is easily fixed by typing the \gls{fin} into the FIN NBR field, and deleted the contents of the ``Name'' field.
\end{quote}

\subsection{Check the Client Field}

\newthought{The patient below} has never been to Pflugerville. Cerner cannot find the patient, because she does not have a Pflugerville encounter.\marginnote[3\baselineskip]{The \textsc{Encounter Search} window will say ``No persons found.''}\marginnote[2\baselineskip]{\textit{See the next page for the solution}}\\

\begin{tikzpicture}
\begin{scope}%[xshift=1.5cm]
    \node[anchor=south west,inner sep=0] (image) at (0,0) {\prettyimage{width=.8\textwidth}{graphics/wrong_patient_location.png}};
    \begin{scope}[x={(image.south east)},y={(image.north west)}]
         \draw [-stealth, line width=3pt, red400] (0.25,1.05) to[out=-90, in=65] (0.09,0.81);
    \end{scope}
\end{scope}
\end{tikzpicture}

\begin{quote}
\newthought{\textbf{Solution:}} In most cases, \textit{Clicking} \btn{graphics/search_button} should bring up the patient. If it doesn't, try this:

\begin{description}
    \paraitem{Close} the \textsc{Encounter Search} Window
    \paraitem{Set} the \textsc{Client} field to (None)
    \paraitem{Retry} the search
\end{description}
\end{quote}
