Log-in by Accession Number is used to log-in samples using the \textit{barcoded} label on the specimen container.

\section{Understanding the Spreadsheet}\label{sec:login_sheet}%

The \textsc{\gls{login} spreadsheet} lists the containers which have been scanned to log-in.
\marginnote[3\baselineskip]{In this example, three containers have been scanned. Container \boldcap{A} and \boldcap{D} of one \gls{accn} and container \boldcap{A} of another.}\\

\prettyimage{width=\textwidth}{graphics/accn_login_example}

\newthought{The rows} are grouped by \glspl{accn} and Containers.

As explained in \checkref{part:accn}{\refptplain{part:accn}}{the \textsc{\glspl{accn}} procedure}, \glspl{accn} can have multiple containers.

\newthought{Under each \gls{accn}} are the containers which have been scanned.\sidenote{In the example above we can see that for \textit{\gls{accn}} \textbf{15-268-000019}, container \boldcap{A}, and \boldcap{D} have been scanned.}

\newthought{Under each container} are the orderables associated with it.\\

\noindent
\begin{tikzpicture}
\begin{scope}
    \node[anchor=south west,inner sep=0] (image) at (0,0) {\prettyimage{width=.8\textwidth}{graphics/accn_login_explain}};
    \begin{scope}[x={(image.south east)},y={(image.north west)}]
    	\draw [-stealth, line width=3pt, purple500] (0.2, 0.85) to[out=270, in=180] (0.44,0.80);
    	\draw [-stealth, line width=3pt, purple500] (0.55, 0.76) to[out=270, in=180] (0.8,0.72);
        \draw [-stealth, line width=3pt, blue500] (0.2, 0.85) to[out=270, in=180] (0.48,0.63);
    	\draw [-stealth, line width=3pt, blue200] (0.55, 0.6) to[out=270, in=180] (0.8,0.54);
        \draw [-stealth, line width=3pt, blue300] (0.55, 0.6) to[out=270, in=180] (0.8,0.46);
    	\draw [-stealth, line width=3pt, blue400] (0.55, 0.6) to[out=270, in=180] (0.8,0.38);
    	\draw [-stealth, line width=3pt, blue500] (0.55, 0.6) to[out=270, in=180] (0.8,0.3);
    \end{scope}
\end{scope}
\end{tikzpicture}%
\marginnote[-6\baselineskip]{This image shows the two containers, and their orders.}

\subsection{Check-boxes}

Each row contains a check-box. By default, they are \faCheckSquareO'd. These can be used to pick which containers are \textsc{Logged-in}.

\begin{description}
  \bolditem{Selected \faCheckSquareO} The orderable or container is selected for an action.\sidenote{\eg{Log-in, Miss\ldots The two main functions.}}
  \bolditem{De-selected \faSquareO} The orderable or container is not select for an action.
  \bolditem{Disabled \disablebox{\faSquareO}} The orderable or container has been \textsc{Logged-In}.
\end{description}

\newthought{After a container} has been \textsc{Logged-in}, the check-box will become \textsc{Disabled} and it can no longer be modified. \marginnote{\refpt{sec:refresh_data} for information on clearing these containers.}

\section{Log-in Accessions Procedure\label{sec:login_procedure}}

\newthought{Open} \gls{login} by clicking the \appicon{specimen_login} icon from the \gls{ab}.

\paragraph{Select} \textsc{Accession} from the  \textsc{Log-in By} options.

\paragraph{Click} \btn{graphics/retrieve.png} to open \textsc{Log-In By ACCESSION}.\\

\prettyimage{width=\textwidth}{graphics/login_by_accession.png}

\paragraph{Check}  that your \textsc{Log-in} location is set appropriately.\\

\prettyimage{width=\textwidth, trim={0 25 150 65 }, clip}{graphics/location.png}

\paragraph{Click} on the \textsc{Accession} entry field.\sidenote{Make sure you see that blinky line thing: |}\marginnote{\info{Skipping this step after changing locations will cause a minor inconvenience.}}\\

\noindent
\begin{tikzpicture}
\begin{scope}[yshift=-.7cm]
    \node[anchor=south west,inner sep=0] (image) at (0,0) {\prettyimage{width=.8\textwidth}{graphics/accession_field}};
    \begin{scope}[x={(image.south east)},y={(image.north west)}]
        \draw [-stealth, line width=3pt, teal400] (1.2, 0.9) to[out=200, in=0] (0.3, 0.7);
    \end{scope}
\end{scope}
\end{tikzpicture}%
\paragraph{Scan} the bar-code of the container.\\

\prettyimage{width=.8\textwidth}{graphics/acc_entered.png}

\importantblock{While, the \gls{accn} can be manually entered, it's not recommended. Accidentally omitting the \gls{cid} will cause uncollected containers to be received in error.\marginnote[-2\baselineskip]{Besides, scanning the tubes is much easier and faster.}}

\newthought{\gls{login} will} update with the container scanned.\\

\prettyimage{width=\textwidth}{graphics/one_container_login}

\importantblock{If the \gls{accn} has multiple containers,\sidenote{Since a Green and Lav have different \glspl{cid}, they are treated individually.} scan them as well. Otherwise, only the scanned container will be logged-in.}

\prettyimage{width=\textwidth}{graphics/two_container}

\subsection{Updating Collect Information}

\importantblock{
This information should be entered at the time of collection.
If it hasn't, the collecting personnel should be notified and instructed to enter it.
}

\newthought{If needed,} the collection information can be updated. This should only be done for outreach samples or if there are extenuating circumstances.

\infoblock{The following instructions will update all the containers.\sidenote{Except the ones which have already been \textsc{Logged-In}. Their check boxes will be disabled \disablebox{\faSquareO}.} If only one value needs to be update \refpt{sec:mod_single}.}
\paragraph{Click} the \textsc{Coll ID Column Header}.\\

\noindent
\begin{tikzpicture}
\begin{scope}
    \node[anchor=south west,inner sep=0] (image) at (0,0) {\prettyimage{width=\linewidth}{graphics/update_collection} };
    \begin{scope}[x={(image.south east)},y={(image.north west)}]
        % \draw [-stealth, line width=3pt, deeppurple400] (0.2, 1.05) to[out=270, in=270] (0.6,0.84);
        % \draw [-stealth, line width=3pt, deeppurple400] (0.2, 1.05) to[out=270, in=270] (0.75,0.84);
        \draw [-stealth, line width=3pt, deeppurple400] (0.2, 1.05) to[out=270, in=270] (0.85,0.84);
    \end{scope}
\end{scope}
\end{tikzpicture}%

\paragraph{Enter} the \textsc{User-Name}\sidenote{This needs to be the Cerner Username. Generic names exist for outside facilities.} of the person who collected the sample.\\

\prettyimage{width=.3\textwidth}{graphics/coll_id}

\paragraph{Click} the \textsc{Coll Time Column Header}.\\

\noindent
\begin{tikzpicture}
\begin{scope}
    \node[anchor=south west,inner sep=0] (image) at (0,0) {\prettyimage{width=\linewidth}{graphics/batch_update_collection_username} };
    \begin{scope}[x={(image.south east)},y={(image.north west)}]
        % \draw [-stealth, line width=3pt, deeppurple400] (0.2, 1.05) to[out=270, in=270] (0.6,0.84);
        \draw [-stealth, line width=3pt, deeppurple400] (0.2, 1.05) to[out=270, in=270] (0.75,0.84);
        % \draw [-stealth, line width=3pt, deeppurple400] (0.2, 1.05) to[out=270, in=270] (0.85,0.84);
    \end{scope}
\end{scope}
\end{tikzpicture}%

\paragraph{Enter} the \textsc{Collection Time} using a 24h format.\sidenote{This needs to be earlier than the \textsc{Receive Time} which defaults to \textsc{Now}.}\\

\prettyimage{width=.3\textwidth}{graphics/coll_time}

\paragraph{Click} the \textsc{Coll Date Column Header}.\\

\noindent
\begin{tikzpicture}
\begin{scope}
    \node[anchor=south west,inner sep=0] (image) at (0,0) {\prettyimage{width=\linewidth}{graphics/batch_update_collection_time} };
    \begin{scope}[x={(image.south east)},y={(image.north west)}]
        \draw [-stealth, line width=3pt, deeppurple400] (0.2, 1.05) to[out=270, in=270] (0.6,0.84);
        % \draw [-stealth, line width=3pt, deeppurple400] (0.2, 1.05) to[out=270, in=270] (0.75,0.84);
        % \draw [-stealth, line width=3pt, deeppurple400] (0.2, 1.05) to[out=270, in=270] (0.85,0.84);
    \end{scope}
\end{scope}
\end{tikzpicture}%

\paragraph{Enter} the \textsc{Collection Date} using a mm/dd/yyyy format.\sidenote{\important{This date defaults to \textsc{Today}. If the sample was collected around midnight, make sure this date is correct.}}\\

\prettyimage{width=.3\textwidth}{graphics/coll_date}

\paragraph{If Needed, Hit} {\faKeyboardO } Ctrl+B to print the labels.
\paragraph{Click} \btn{graphics/login_btn} to Log-in\sidenote{\hotkey{\boldcap{Ctrl+L} will also work.}}\marginnote{\info{If you get a pop-up asking about ``Separate Collections'' \refpt{sec:sep_collection}}.}\\

\prettyimage{width=\textwidth}{graphics/logged_in}

\newthought{At this point}, the samples have been \textsc{Logged-in}.  Those containers will remain in the window until it is refreshed.\marginnote{\refpt{sec:refresh_data} for information on clearing these containers.}

\subsection{The Miss Button}

Please ignore this button. As of right now it doesn't fit into our work flow.