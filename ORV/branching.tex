From \acrlong{orv}, orders can be opened in other applications. This allows us to quickly and easily switch between tasks.

\newthought{Most of the} branching can be done using the icons from the tool-bar.
\paragraph{Select} the order to open in the branched application.
\paragraph{Click} the appropriate Icon from the toolbar.\sidenote{\info{Hovering over the icons will display their name.}}

\newthought{\gls{ci}:} allows us to quickly view tracking information on a sample.\\

\noindent
\begin{tikzpicture}
\begin{scope}
    \node[anchor=south west,inner sep=0] (image) at (0,0) {\prettyimage{width=\textwidth}{graphics/orv_toolbar2}};
    \begin{scope}[x={(image.south east)},y={(image.north west)}]
        \draw[deeppurple400, rounded corners, line width=3] (0.44, 0.1)rectangle (0.53, 0.98);
    \end{scope}
\end{scope}
\end{tikzpicture}%

\newthought{\gls{are}:} allows us to enter, perform or verify results on a selected order.\\

\noindent
\begin{tikzpicture}
\begin{scope}
    \node[anchor=south west,inner sep=0] (image) at (0,0) {\prettyimage{width=\textwidth}{graphics/orv_toolbar2}};
    \begin{scope}[x={(image.south east)},y={(image.north west)}]
        \draw[teal400, rounded corners, line width=3] (0.51, 0.1)rectangle (0.6, 0.98);
    \end{scope}
\end{scope}
\end{tikzpicture}%

\newthought{\gls{login}:} allows us to log-in any samples that may have been missed, or logged into another location.\\

\noindent
\begin{tikzpicture}
\begin{scope}
    \node[anchor=south west,inner sep=0] (image) at (0,0) {\prettyimage{width=\textwidth}{graphics/orv_toolbar2}};
    \begin{scope}[x={(image.south east)},y={(image.north west)}]
        \draw[indigo400, rounded corners, line width=3] (0.73, 0.1)rectangle (0.82, 0.98);
    \end{scope}
\end{scope}
\end{tikzpicture}%

