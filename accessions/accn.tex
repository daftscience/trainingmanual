\chapter{Introduction}\label{ch:accn_intro}
In Cerner, an \gls{accn} is the unique number assigned to a set of orders with the same sample type\sidenote{\eg{blood, urine, body fluid, \etc}}, and collection time.

This means, when the emergency department sends a ``Rainbow'' all the orders for those tubes will be on a single \gls{accn}.\\

\newthought{\Glspl{accn}} are made up of four parts:
\begin{description}
    \bolditem{Julian Year} 2-digit number representing the current year.
    \bolditem{Julian Day} 3-digit number representing the day of the year the sample was, or will be, collected. (001-365)\sidenote{\info{It's alright if they don't match. Samples collected around midnight will often have a different Julian date than collect date.}}
    \bolditem{Group Number} A unique number to differentiate \glspl{accn}.
    \bolditem{Container Identifier} A \textit{letter} used to differentiate containers attached to an \gls{accn}
\end{description}

\vspace{2em}
\noindent
\begin{minipage}{\textwidth}
\centering
    \sffamily
    \begin{tikzpicture}
        {\Large
        \node [anchor=west] (pif) at (-0.1, 1.5) {\LARGE
                    \color{deeporange800}{Julian Year} \hspace{2em}
                    \color{cyan800}{Group Number}};
        \node [anchor=west] (sp) at (0.0,-2) {\LARGE
                    \color{purple800}{Julian Day} \hspace{1em}
                    \color{indigo800}{Container Id}};
        }
        \node[anchor=west,inner sep=0] (image) at (0,0) {
                \Huge{
                        {\color{deeporange600}15}-%
                        {\color{purple600}123}-%
                        {\color{cyan600}123456}%
                        {\color{indigo600}A}%
                        }};
        \draw [-stealth, line width=3pt, deeporange400] (1.4,  1.3)    to[out=-90, in=80] (0.6, 0.3);
        \draw [-stealth, line width=3pt, cyan400]            (4.7,  1.3)    to[out=-90, in=60] (4, 0.3);
        \draw [-stealth, line width=3pt, purple400]         (1.4, -1.8)   to[out=90, in=-90] (2, -0.3);
        \draw [-stealth, line width=3pt, indigo400]         (4.1, -1.8)   to[out=90, in=-90] (6, -0.3);
    \end{tikzpicture}%
\end{minipage}

\newthought{The group number} will reset at midnight. Since the date is built into the \gls{accn}, no two collections will ever have the same \gls{accn}.

\clearpage
\section{Container Identifiers}\label{sec:accn_cid}

The \gls{cid} gives Cerner the ability to group \textit{every} container from a collection onto a single \gls{accn}, while still allowing the instruments to differentiate containers.

\newthought{Example:}

\begin{quote}
An order is placed for a \textbf{CBC}, \textbf{Basic Panel}, \textbf{Lithium} and a \textbf{Troponin}.
Cerner will create one \gls{accn} with three containers.
    \begin{itemize}
        \item 15-123-000001A - \textbf{CBC}
        \item 15-123-000001B - \textbf{Basic Panel}, \textbf{Troponin}(Li-Heparin)
        \item 15-123-000001C - \textbf{Lithium}(SST)
    \end{itemize}
When container ``B,'' is loaded on to the instrument, the only orders it will see are the \textbf{Basic Panel} and \textbf{Troponin}.
\end{quote}

\importantblock{Check that the correct label is placed on each of the containers. If the SST label is placed on the Li-Heparin tube, the instruments will run a lithium on the Li-Heparin container.}

\section{Advantages}

By assigning one \gls{accn} to an entire collection, specimen tracking becomes much easier.

With \Gls{ci}, it's possible to see every container on an \gls{accn}, and each of the tests associated with those containers.

Sites with automation lines will be able to load extra containers onto the line. Those containers will be spun and immediately filed.

\section{Downtime Accessions}
\warning{WORK IN PROGRESS}\\
In the event that Cerner is unable to generate \gls{accn} (\eg{\glspl{down}},) \gls{down} \glspl{accn} will need to be manually assigned using \gls{doe}.\sidenote{\checkref{part:downtime}{\refpt{part:downtime}}{\textit{Refer to the Downtime Procedure.}}}

Pre-printed \gls{down} labels have \glspl{accn} similar to normal \glspl{accn}. The main difference is that the Julian day is replaced with a number greater than 366.\sidenote{\info{We can safely use these numbers because normal \glspl{accn} can't go that high. If they do, then it's been a long year.}}

\newthought{Example of a} \gls{down} \gls{accn}:\\

\begin{tikzpicture}
    \sffamily
        \node[anchor=west,inner sep=0] (image) at (0,0) {\LARGE{{\color{deeppurple500}15-400-000001}}};
    \end{tikzpicture}%
% \end{minipage}

\chapter{Tips and Tricks}\label{ch:accn_tips}
While on the surface, entering \glspl{accn} in Cerner applications is fairly simple, this section will go over some of the details.

\section{Container ID}
In most Applications, Cerner will ignore the \gls{cid}%
\sidenote{This means Cerner will pull up information on the entire order}. There are two exceptions \ie{\gls{ci} and \gls{login}}.


\newthought{With a }\Gls{cid}: 15-123-000001\boldcap{B} \marginnote[3\baselineskip]{\info{Only the entered container appears.}}\\

\prettyimage{width=\textwidth, trim={0 150pt 0 0}, clip}{graphics/with_id.png}

\newthought{Without a }\Gls{cid}: 15-123-000001\marginnote[3\baselineskip]{\info{Both containers on the \gls{accn} appear.}}\\

\prettyimage{width=\textwidth, trim={0 150pt 0 0}, clip}{graphics/with_cid.png}
\importantblock{\appicon{specimen_login} \gls{login} also behaves this way. Meaning, if the \gls{accn} is entered without the \gls{cid}, it's possible to log-in containers which have not been sent to the lab.}\marginnote[-2\baselineskip]{\ldots And that is why you always use the Barcode Scanners.}

\pagebreak
\section{Entering Accession Numbers\label{sec:accn_tips_entering}}
If you haven't noticed, the accession numbers are a bit long. Luckily we have some shortcuts.

\newthought{When entering an Accn} we can auto-fill most of the information.
This will allow us to quickly enter \glspl{accn}.\\

\subsection{Fast Entry}
\noindent
\begin{minipage}{.5\textwidth}
    \paragraph{Click} on the accession field.
\end{minipage}\marginnote{If it's not empty, delete the stuff in it.}%
\begin{minipage}{.1\textwidth}
{ \color{white} hi}
\end{minipage}%
\begin{minipage}{.5\textwidth}
    \prettyimage{width=.6\textwidth}{graphics/accnfield_blank.png}
\end{minipage}\\

\noindent
\begin{minipage}{.5\textwidth}
    \paragraph{Hit} {\faKeyboardO} \boldcap{Tab} on the keyboard.
    % \paragraph{Hit} the \textbf{Tab} key \faKeyboardO
\end{minipage}\marginnote{The current year just auto-filled}%
\begin{minipage}{.1\textwidth}
{ \color{white} hi}
\end{minipage}%
\begin{minipage}{.5\textwidth}
    \prettyimage{width=.6\textwidth}{graphics/accnfield_year.png}
\end{minipage}\\

\noindent
\begin{minipage}{.5\textwidth}
    \paragraph{Hit} {\faKeyboardO} \boldcap{Tab} on the keyboard.
\end{minipage}\marginnote{You guessed it, the current \\day just auto-filled.}%
\begin{minipage}{.1\textwidth}
{ \color{white} hi}
\end{minipage}%
\begin{minipage}{.5\textwidth}
    \prettyimage{width=.6\textwidth}{graphics/accnfield_day.png}
\end{minipage}

\noindent
\begin{minipage}{.5\textwidth}
    \paragraph{Hit} {\faKeyboardO} \boldcap{Tab} on the keyboard.
\end{minipage}%
\begin{minipage}{.1\textwidth}
{ \color{white} hi}
\end{minipage}%
\begin{minipage}{.5\textwidth}
    \prettyimage{width=.6\textwidth}{graphics/accnfield_done.png}
\end{minipage}\marginnote{\faMagic The whole \gls{accn} was pulled up using the last 3 digits.}


\subsection{Faster Entry}
\newthought{If the \gls{accn}} has today's date, you can actually skip the first two tabs.\\

\noindent
\begin{minipage}{.5\textwidth}
\paragraph{Click} on the accession field.
\end{minipage}\marginnote{If it's not empty, delete the stuff in it.}%
\begin{minipage}{.1\textwidth}
{ \color{white} hi}
\end{minipage}%
\begin{minipage}{.5\textwidth}
\prettyimage{width=.6\textwidth}{graphics/accnfield_blank.png}
\end{minipage}

\noindent
\begin{minipage}{.5\textwidth}
\paragraph{Hit} {\faKeyboardO} \boldcap{Enter}.
\end{minipage}\marginnote{The current year and day just auto-filled. Pretty nifty.}%
\begin{minipage}{.1\textwidth}
{ \color{white} hi}
\end{minipage}%
\begin{minipage}{.5\textwidth}
\prettyimage{width=.6\textwidth}{graphics/accnfield_day.png}
\end{minipage}

% \noindent
% \begin{minipage}{.5\textwidth}
%     \paragraph{Hit} the \textbf{Tab} key \faKeyboardO
% \end{minipage}\marginnote{You guessed it, the current \\day just auto-filled.}%
% \begin{minipage}{.5\textwidth}
%     \prettyimage{width=.6\textwidth}{graphics/accnfield_day.png}
% \end{minipage}

\noindent
\begin{minipage}{.5\textwidth}
    \paragraph{Hit} {\faKeyboardO} \boldcap{Tab}.
\end{minipage}%
\begin{minipage}{.1\textwidth}
{ \color{white} hi}
\end{minipage}%
\begin{minipage}{.5\textwidth}
    \prettyimage{width=.6\textwidth}{graphics/accnfield_done.png}
\end{minipage}