\RequirePackage{etoolbox}
\providetoggle{debug}
\providetoggle{offset}
\providetoggle{showlayout}
\providetoggle{isdraft}
\providetoggle{ishandout}

\providetoggle{onlineversion}

\providetoggle{showAnswers}
\providetoggle{justActivities}
\providetoggle{includeActivities}
\providetoggle{instruction}


%%%%%%%%%%%%%%%%%%%%%%%%%%%%%%
\settoggle{debug}{false}
\settoggle{offset}{false}
\settoggle{isdraft}{false}
\settoggle{showlayout}{false}
\settoggle{ishandout}{false}
\settoggle{onlineversion}{false}
%%%%%%%%%%%%%%%%%%%%%%%%%%%%%%


%%%%%%%%%%%%%%%%%%%%%%%%%%%%%%
\settoggle{justActivities}{false}
\settoggle{showAnswers}{false}
\settoggle{includeActivities}{true}
\settoggle{instruction}{true}



\iftoggle{justActivities}{
    \settoggle{onlineversion}{true}
    \settoggle{includeActivities}{false}
  }
%%%%%%%%%%%%%%%%%%%%%%%%%%%%%%



\newcommand{\warncolor}{\color{red900}}
\newcommand{\impcolor}{\color{cyan700}}
\newcommand{\refcolor}{\color{deeppurple700}}
\newcommand{\hotcolor}{\color{indigo700}}
\newcommand{\infocolor}{\color{bluegrey800}}

\newcommand{\answercolor}{\color{indigo900}}
\newcommand{\answerRefColor}{\color{cyan900}}
% \newcommand{\questioncolor}{\color{indigo900}}
\newcommand{\answerkeycolor}{\color{red900}}


\newcommand{\titlepagecolor}{bluegrey50}
\newcommand{\chaptercolor}{\color{indigo900}}
\newcommand{\sectioncolor}{\color{indigo900}}
\newcommand{\subsectioncolor}{\color{indigo800}}


\iftoggle{debug}{
	\IfFileExists{../tufte-book.cls}{\documentclass{../tufte-book}}{\documentclass{tufte-book}}
	\usepackage[none]{hyphenat}
}{%
  \iftoggle{onlineversion}{
    \IfFileExists{../tufte-book.cls}{\documentclass[oneside,openany, marginals, justified]{../tufte-book}}{%
      \documentclass[oneside, openany, marginals, justified]{tufte-book}}{}
  }{
    \IfFileExists{../tufte-book.cls}{\documentclass[marginals, justified]{../tufte-book}}{%
      \documentclass[marginals, justified]{tufte-book}}{}
  }
}




\iftoggle{offset}{
    \setlength{\evensidemargin}{-0.3125in}
    \setlength{\oddsidemargin}{0.3in}
}
\iftoggle{showlayout}{
  \usepackage[texcoord,gridunit=cm]{showframe}
}
\iftoggle{isdraft}{
  \usepackage{draftwatermark}
  \SetWatermarkText{DRAFT}
  \SetWatermarkScale{1}
}

% \setsidenotefont{\color{black}\footnotesize}
% \setmarginnotefont{\color{black}\footnotesize}



\RequirePackage{fontawesome}

\let\oldkb\faKeyboardO
\renewcommand{\faKeyboardO}{\hotcolor\oldkb}

\newcommand{\activityicon}{\faDesktop}

\newcommand{\pencilParagraph}[1]{%
\renewcommand{\activityicon}{\faPencil}%
\paragraph{#1}
\renewcommand{\activityicon}{\faDesktop}}

\usepackage{pgffor, ifthen}
\newcommand{\notes}[3][\empty]{%
    \noindent {\faStickyNoteO} \boldcap{ Notes}:\vspace{10pt}\\
    \foreach \n in {1,...,#2}{%
        \ifthenelse{\equal{#1}{\empty}}
            {\rule{#3}{0.5pt}\\}
            {\rule{#3}{0.5pt}\vspace{#1}\\}
        }
}

% Better horizontal rules in tables
\usepackage{booktabs}
\usepackage{longtable}
\AtBeginDocument {\sloppy}

\newcommand{\smartinput}[1]{
  \IfFileExists{#1}{\input{#1}}{\input{../#1}}
}


% Image things
\usepackage{graphicx}
\graphicspath{{graphics/}}
\setkeys{Gin}{width=\linewidth,totalheight=\textheight,keepaspectratio}
\usepackage{tikz}
\usepackage{forest}
\usetikzlibrary{trees}
\usepackage{tikz-qtree}
\usetikzlibrary{shadows.blur}
\usetikzlibrary{shapes,arrows}
\usepackage[export]{adjustbox}
\usepackage[framemethod=TikZ]{mdframed}
\usepackage{tabularx}


% Define block styles
\tikzstyle{decision} = [diamond, draw, fill=blue!20,
    text width=4.5em, text badly centered, node distance=3cm, inner sep=0pt]
\tikzstyle{block} = [rectangle, draw, fill=blue!20,
    text width=5em, text centered, rounded corners, minimum height=4em]
\tikzstyle{line} = [draw, -latex']
\tikzstyle{cloud} = [draw, ellipse,fill=red!20, node distance=3cm,
    minimum height=2em]

\forestset{
  dir tree/.style={
    for tree={
      parent anchor=south west,
      child anchor=west,
      anchor=mid west,
      inner ysep=1pt,
      grow'=0,
      align=left,
      edge path={
        \noexpand\path [draw, \forestoption{edge}] (!u.parent anchor) ++(1em,0) |- (.child anchor)\forestoption{edge label};
      },
      font=\sffamily,
      if n children=0{}{
        delay={
          prepend={[,phantom, calign with current]}
        }
      },
      fit=band,
      before computing xy={
        l=2em
      }
    },
  }
}



\newcommand{\treeitem}[2]{%
{\includegraphics[height=1em]{graphics/routing_icons/#1} {\scshape #2}}}

\usepackage{tcolorbox}
 \tcbuselibrary{most}
\newtcbox{\disablebox}{colback=bluegrey200, nobeforeafter, size=tight}
\newcommand*{\ClipSep}{2pt}%



\newcommand{\framedimg}[2]{%
\begin{tikzpicture}
\node [inner sep=0pt] at (0,0) {\includegraphics[#1]{#2}};
\draw [cyan800, rounded corners=\ClipSep, line width=\ClipSep/2]
    (current bounding box.north west) --
    (current bounding box.north east) --
    (current bounding box.south east) --
    (current bounding box.south west) -- cycle
    ;\end{tikzpicture}}

\newcommand{\framedicon}[2]{%
\begin{tikzpicture}
\node [inner sep=0pt] at (0,0) {\includegraphics[#1]{#2}};
\draw [bluegrey500, rounded corners=.5pt, line width=0]
    (current bounding box.north west) --
    (current bounding box.north east) --
    (current bounding box.south east) --
    (current bounding box.south west) -- cycle
    ;\end{tikzpicture}}


\newcommand{\prettyimage}[3][]{%
\noindent
\tcbox[%
      enhanced,
      boxsep=0pt, top=0pt, bottom=0pt, left=0pt, right=0pt,%
      boxrule=0pt,
      drop lifted shadow,
      clip upper,
      colback=teal500,
      nobeforeafter]{%
\framedimg{#2}{#3}}
}

% The fancyvrb package lets us customize the formatting of verbatim
% environments.  We use a slightly smaller font.
\usepackage{fancyvrb}
\fvset{fontsize=\normalsize}

%%
% Prints a trailing space in a smart way.
\usepackage{xspace}

\usepackage{mathtools}


% Macros for typesetting the documentation
\newcommand{\hlred}[1]{\textcolor{Maroon}{#1}}% prints in red
\newcommand{\hangleft}[1]{\makebox[0pt][r]{#1}}
\newcommand{\hairsp}{\hspace{1pt}}% hair space
\newcommand{\hquad}{\hskip0.5em\relax}% half quad space
\newcommand{\TODO}{\textcolor{red}{\bf TODO!}\xspace}
\newcommand{\ie}{\textit{i.\hairsp{}e.}\xspace}
\newcommand{\eg}{\textit{e.\hairsp{}g.}\xspace}
\newcommand{\na}{\quad--}% used in tables for N/A cells

\renewcommand{\ie}[1]{\textit{i.\hairsp{}e.\xspace #1}} % Command to print i.e.
\renewcommand{\eg}[1]{\textit{e.\hairsp{}g.\xspace #1}} % Command to print i.e.
\newcommand{\etc}{\textit{etc\ldots}}
\newcommand{\abspace}{\textbf{--(Space)--} }

% \newcommand{\bolditem}[1]{\item \textbf{\scshape #1:}}

% Prints the month name (e.g., January) and the year (e.g., 2008)
\newcommand{\monthyear}{%
  \ifcase\month\or January\or February\or March\or April\or May\or June\or
  July\or August\or September\or October\or November\or
  December\fi\space\number\year
}

% Prints an epigraph and speaker in sans serif, all-caps type.
\newcommand{\openepigraph}[2]{%
  %\sffamily\fontsize{14}{16}\selectfont
  \begin{fullwidth}
  \sffamily\large
  \begin{doublespace}
  \noindent\allcaps{#1}\\% epigraph
  \noindent\allcaps{#2}% author
  \end{doublespace}
  \end{fullwidth}
}

% Generates the index
\usepackage{makeidx}
\makeindex

\newcommand{\blankpage}{\newpage\hbox{}\thispagestyle{empty}\newpage} % Command to insert a blank page

% This makes folder paths work better
\usepackage{chapterfolder}
\let\includegraphicsWithoutCF\includegraphics
\renewcommand{\includegraphics}[2][]{\includegraphicsWithoutCF[#1]{\cfcurrentfolder#2}}

\usepackage{tabto}

\usepackage[nopostdot, xindy, toc, acronym, nohypertypes={acronym,notation}]{glossaries}
\makeglossaries

\newglossarystyle{acrstyle}{%
 \setglossarystyle{altlist}%
 \renewcommand*{\glossentry}[2]{%
    \NumTabs{6}
   \item \tab \LARGE\glsentryitem{##1}\boldcap{\glstarget{##1}{\glossentryname{##1}}}%
   \tab \scshape{\glossentrydesc{##1}}
 }%
}


\newglossarystyle{mystyle}{%
 \setglossarystyle{treegroup}%
 \renewcommand*{\glsgroupheading}[1]{{\huge{##1}}\\}%
 \renewcommand*{\glossentry}[2]{%
   \glsentryitem{##1}\boldcap{\glstarget{##1}{\glossentryname{##1}}}%
   \begin{quote}\glossentrydesc{##1}\end{quote}
 }%
}

\newcommand{\button}[1]{\includegraphics[height=1.5em]{#1}}
\newcommand{\btn}[1]{\tcbox[%
       enhanced,
       boxsep=0pt, top=0pt, bottom=0pt, left=0pt, right=0pt,%
       boxrule=0pt,
       drop fuzzy shadow,
       clip upper,
       colback=black!75!white,
       nobeforeafter]{%
 \framedimg{height=1.5em}{#1}}}

% \newcommand{\btn}[1]{\includegraphics[height=1.5em, cframe={bluegrey300} {1pt}]{#1}}
% \newcommand{\btn}[1]{\framedicon{height=1.5em}{#1}}
\newcommand{\gfx}{graphics/apps/}
% \newcommand{\appicon}[1]{{\includegraphics[height=1.5em, cframe={bluegrey300} {1pt}]{\gfx #1}}}
\newcommand{\appicon}[1]{\tcbox[%
       enhanced,
       arc=3pt,
       borderline={2pt}{0pt}{bluegrey500},
       drop fuzzy shadow,
       size=tight,
       nobeforeafter]{%
 \includegraphics[height=1.5em]{\gfx #1}} }





\iftoggle{debug}{
  \DeclareDocumentCommand{\textls}{O{1}O{2}m}{#3}
  \renewcommand{\allcapsspacing}[1]{\textls[1][2]{#1}}
  \renewcommand{\smallcapsspacing}[1]{\textls[2][1.4]{#1}}
  \renewcommand{\allcaps}[1]{\textls[10]{\MakeTextUppercase{#1}}}
  \renewcommand{\smallcaps}[1]{\smallcapsspacing{\scshape\MakeTextLowercase{#1}}}
  \renewcommand{\textsc}[1]{\smallcapsspacing{\textsmallcaps{#1}}}
}{
  \DeclareDocumentCommand{\textls}{O{1}O{2}m}{%
    \begingroup\addfontfeatures{LetterSpace=#1, WordSpace=#2}#3\endgroup}
  \renewcommand{\allcapsspacing}[1]{\textls[1][2]{#1}}
  \renewcommand{\smallcapsspacing}[1]{\textls[2][1.4]{#1}}

  \renewcommand{\allcaps}[1]{\textls[10]{\MakeTextUppercase{#1}}}
  \renewcommand{\smallcaps}[1]{\smallcapsspacing{\scshape\MakeTextLowercase{#1}}}
  \renewcommand{\textsc}[1]{\smallcapsspacing{\textsmallcaps{#1}}}
}



\usepackage{fontspec}

% Some commands to standardize margin notes
\newcommand{\tip}[1]{\textit{\textbf{Tip}: #1}}
\newcommand{\hotkey}[1]{\hotcolor\faKeyboardO \textit{\textbf{ Hotkey: } #1}}
\newcommand{\warning}[1]{\textit{\warncolor\faWarning \textbf{ WARNING: } #1}}
\newcommand{\important}[1]{{\impcolor\faWarning \textbf{ IMPORTANT: } #1}}
% \newcommand{\important}[1]{\faWarning \textbf{ IMPORTANT: } \textit{#1}}
\newcommand{\note}[1]{\textit{\textbf{Note: } #1}}

\newcommand{\info}[1]{{\infocolor\textbf{\faInfo nfo:} #1}}
\newcommand{\hint}[1]{{\infocolor\textbf{Hint:} #1}}

% \newcommand{\importantblock}[1]{\begin{quote}\scshape\textbf{\faExclamationTriangle { IMPORTANT: }} #1 \end{quote}}
% \newcommand{\importantblock}[1]{\begin{quote} \important{ #1 } \end{quote}}
% \newcommand{\importantblock}[1]{\begin{quote} #1 \end{quote}}
\newcommand{\importantblock}[1]{\begin{quote} \important{#1} \end{quote}}
\newcommand{\warnblock}[1]{\begin{quote} \warning{#1} \end{quote}}
% \newcommand{\warnblock}[1]{\begin{quote}\scshape\textbf{\faExclamationTriangle { WARNING}:} #1 \end{quote}}

\newcommand{\infoblock}[1]{\begin{quote}\info{#1}\end{quote}}
\newcommand{\hintblock}[1]{\begin{quote}\hint{#1}\end{quote}}
\setmainfont{TeX Gyre Pagella}
\setsansfont{TeX Gyre Heros}[Scale=MatchUppercase]


\newcommand{\boldcap}[1]{\textbf{\textsc{#1}}}
\newcommand{\bolditem}[1]{\item \boldcap{#1:}}
\newcommand{\paraitem}[1]{\item{\itshape\textls[15]{{\faDesktop} #1}  \hspace*{.5em}} }
\newcommand{\trouble}[1]{\faWrench \boldcap{ Troubleshooting: }#1 }




% \newcommand{\fontset}{Lato}
\newcommand{\fontset}{SourceSansPro}
% \newcommand{\fontset}{Roboto}
% \newcommand{\fontset}{Delicious}
% \newcommand{\fontset}{FontinSans}
% \newcommand{\fontset}{ArchivoNarrow}
\newcommand{\fontfolder}{fonts/}

\newcommand{\boldface}{-Bold}
\newcommand{\italicface}{-Italic}
\newcommand{\regularface}{-Regular}
\newcommand{\bolditalicface}{-BoldItalic}
\newcommand{\smallcapface}{-SmallCaps}

\ifthenelse{\equal{\fontset}{SourceSansPro}}{
    \renewcommand{\boldface}{-Semibold}
    \renewcommand{\bolditalicface}{-SemiboldItalic}
}{}

\IfFileExists{\fontfolder\fontset/\fontset-Regular.ttf}{%
}{%
    \renewcommand{\fontfolder}{../fonts/}
}
\setmainfont{\fontset}[
    Path = \fontfolder\fontset/ ,
    UprightFont = *\regularface ,
    ItalicFont = *\italicface ,
    BoldFont = *\boldface ,
    BoldItalicFont = *\bolditalicface ,
    % SmallCapsFont = *\smallcapface,
  ]

\smartinput{terms.tex}
\smartinput{material_colors}
\smartinput{layout.tex}

\newcommand*{\IsAlphaCharacter}[3]{
  \int_compare:nTF{ (`#1 >= `a && `#1 <=`z) || (`#1>= `A && `#1 <= `Z) } {#1 #2}{#1 #3}}

% Fix part references
\makeatletter
\let\oldpart\part
\def\part#1{\def\@currentlabelname{#1}\oldpart{#1}}
\makeatother







\IfFileExists{graphics/seton/seton_Healthcarefamily_alt_cmyk.jpg}{%
  \newsavebox{\titleimage}
  \savebox{\titleimage}{\includegraphics[height=7\baselineskip]{graphics/seton/seton_Healthcarefamily_alt_cmyk.png}}
}{
  \newsavebox{\titleimage}
  \savebox{\titleimage}{\includegraphics[height=7\baselineskip]{../graphics/seton/seton_Healthcarefamily_alt_cmyk.png}}
}


\usepackage{nameref}
\usepackage{cleveref}

% \makeatletter
% \newcommand{\iflabelexists}[3]{\@ifundefined{r@#1}{#3}{#2}}
% \makeatother
\newcommand{\checkref}[3]{\ifcsundef{r@#1}{{\refcolor {\reficon} #3}}{#2}}
\newcommand{\checkrefbookless}[3]{\ifcsundef{r@#1}{{\refcolor #3}}{#2}}

\creflabelformat{part}{#1#2#3}
\creflabelformat{ch}{#1#2#3}

\newcommand{\reficon}{\refcolor\faBook}


\newcommand{\refptplain}[1]{{\refcolor {\reficon} {\boldcap{\nameref{#1}} \textit{pg. \pageref{#1}}}}}

% \crefname{part}{Part}{Parts}
% \crefname{ch}{Chapter}{Chapter}
\newcommand{\refch}[1]{{\refcolor {\reficon} Refer to: {\scshape\nameref{#1} \textit{pg. \pageref{#1}}}}}
\newcommand{\refpt}[1]{{\refcolor {\reficon} Refer to: {\boldcap{\nameref{#1}} \textit{pg. \pageref{#1}}}}}
\newcommand{\refptch}[2]{{\refcolor {\reficon} Refer to: {\scshape\textbf{\nameref{#1}}: \nameref{#2} \textit{pg. \pageref{#2}}}}}


\newcommand{\refptapp}[2]{{\refcolor {\reficon} Refer to: {\miniappicon{#2}\boldcap{\nameref{#1}} \textit{pg. \pageref{#1}}}}}
\newcommand{\refptchapp}[3]{{\refcolor {\reficon} Refer to: {\miniappicon{#3}\scshape\textbf{\nameref{#1}}: \nameref{#2} \textit{pg. \pageref{#2}}}}}

\newcommand{\reftable}[1]{{\refcolor {\reficon} Refer to Table: {\boldcap{\nameref{#1}} \textit{pg. \pageref{#1}}}}}


\newcommand{\accntip}{\sidenote{{\checkref{sec:accn_tips_entering}{\refptch{part:accn}{sec:accn_tips_entering} for some nifty tips}{See the \textsc{\gls{accn}} handout for more information}}.}}


% app
\newcommand{\appitem}[2]{\item \appicon{#1}{{\Large{\boldcap{ \gls{#2}}}}}}

\newcommand{\checkdemo}{\checkrefbookless{part:appendix}{%
  \section{Customizing Demographics}
    \refptch{part:appendix}{ch:demographics} for detailed instructions on modifying the demographic information.}{%
    \cfchapter{Customizing Demographics\label{ch:handout_demo}}{../demographics}{custom_demos.tex}}}

\newcommand{\addrouting}{%
\IfFileExists{./routing/routing.tex}{}{
\chapter{Routing\label{ch:appended_routing}}
\IfFileExists{bhrouting.tex}{\addcontentsline{toc}{chapter}{About Routing}}
\section{What is Routing?}

When Accession Numbers are assigned, Cerner will determine a \textsc{Service Resource}\sidenote{Testing location.} for the containers. The \textsc{Service Resource} is based on the \textsc{Procedure},\sidenote{\eg{CBC, Basic Panel, HbA\textsubscript{1c}, \etc}} \ and \textsc{Collection Location} of the container.


\newthought{This gives Cerner} the ability to update the status of a sample automatically based on the last place it was \textsc{Logged-in.}

\begin{quote}
    \boldcap{For example: } Let's say a CBC and a HbA\textsubscript{1c} are sent to \textsc{SMCA Laboratory}.

    When they're \textsc{Logged-in} to \textsc{SMC Login}, the CBC have an \textsc{In-Lab} status,\sidenote{Meaning, it's available to be tested.} while the HbA\textsubscript{1c} will have \textsc{Pending} status.\sidenote{Meaning, it's been collected but hasn't been sent yet.}
\end{quote}


\newthought{In addition to tracking,} the \textsc{Service Resources} is used in \gls{pi} as the \textsc{Test Site}.

\section{How Does It Work?}

\subsection{Instruments and Benches and Sections oh my!}

Every test is \boldcap{Routed} to either an \treeitem{instrument}{Instrument} or \treeitem{bench}{Bench}.\\

\treeitem{instrument}{Instruments} are interfaced instruments.\sidenote{They can send the results to Cerner.}

\treeitem{bench}{Benches} are where manual tests are performed.

 Since there are so many instruments\sidenote{We should start a band!} and benches we need a way to organize them. To do this we have \treeitem{section}{Sections} and \treeitem{subsection}{Subsections}.\\

 \treeitem{section}{Sections} are departments in the lab.\sidenote{Heme, Chem, UA, \etc}

 \treeitem{subsection}{Subsections} are areas of the sections.\sidenote{Manual Chemistry, Tegs, Miscellaneous Micro, \etc}


\subsection{All That To Say:}

Tests are routed to a \treeitem{bench}{Bench} or \treeitem{instrument}{Instrument}.

\noindent
\treeitem{bench}{Benches} and \treeitem{instrument}{Instruments} belong to a \treeitem{subsection}{Subsection}.

\noindent
\treeitem{subsection}{Subsections} belongs to a \treeitem{section}{Section}.

% \section{Example of Routing}
% {\setstretch{0.5}\scriptsize\scshape
%     \cfinput{../routing/smcrouting.tex}
% }

% This chapter contains a breakdown of the Seton Network Laboratories and their routing.

% \section{Instruments and Benches}

% Every test is \boldcap{Routed} to either an \instrument or \bench.\\

% \instruments are interfaced instruments.\sidenote{They can send the results to Cerner.}

% \benches are where manual tests are performed.

% \section{Sections and Subsections}

%  Since there are so many instruments\sidenote{We should start a band!} and benches we need a way to organize them. To do this we have \sects and \subsects.\\

% \sects are departments in the lab.\sidenote{Heme, Chem, UA, \etc}

% \subsects are areas of the sections.\sidenote{Manual Chemistry, Tegs, Miscellaneous Micro, \etc}


% \subsection{All That To Say:}

% Tests are routed to a \bench or \instrument.

% \noindent
% \benches and \instruments belong to a \subsect.

% \noindent
% \subsects belongs to a \sect.
% \clearpage


{\setstretch{0.5}\scriptsize\scshape
    \tikzset{
      dirtree/.style={
        growth function=\dirtree@growth,
        every node/.style={anchor=north},
        every child node/.style={anchor=west},
        edge from parent path={(\tikzparentnode\tikzparentanchor) |- (\tikzchildnode\tikzchildanchor)},
      }
    }

    \IfFileExists{bhrouting.tex}{%
        \addcontentsline{toc}{chapter}{Routing Maps}
        \cfsection{Brackenridge}{./}{bhrouting.tex}
        \cfsection{Dell Children's}{./}{dcrouting.tex}
        \cfsection{Edgar B. Davis}{./}{sebdrouting.tex}
        \cfsection{Hays}{./}{shcrouting.tex}
        \cfsection{Highland Lakes}{./}{shlrouting.tex}
        \cfsection{Northwest}{./}{snwrouting.tex}
        \cfsection{Southwest}{./}{sswrouting.tex}
        \cfsection{Williamson}{./}{swcrouting.tex}
        \cfsection{Seton Medical Center}{./}{smcrouting.tex}
    }{%
        \label{sec:routing_maps}
        \cfsection{Brackenridge}{../routing}{bhrouting.tex}
        \cfsection{Dell Children's}{../routing}{dcrouting.tex}
        \cfsection{Edgar B. Davis}{../routing}{sebdrouting.tex}
        \cfsection{Hays}{../routing}{shcrouting.tex}
        \cfsection{Highland Lakes}{../routing}{shlrouting.tex}
        \cfsection{Northwest}{../routing}{snwrouting.tex}
        \cfsection{Southwest}{../routing}{sswrouting.tex}
        \cfsection{Williamson}{../routing}{swcrouting.tex}
        \cfsection{Seton Medical Center}{../routing}{smcrouting.tex}
    }
}
% \cfchapter{Routing\label{ch:appended_routing}}{../routing}{routing.tex}
}%
% \checkref{part:appendix}{\refptch{part:appendix}{ch:routing} for a detailed diagram of Seton's routing.}{\IfFileExists{bhrouting.tex}{\addcontentsline{toc}{chapter}{About Routing}}
\section{What is Routing?}

When Accession Numbers are assigned, Cerner will determine a \textsc{Service Resource}\sidenote{Testing location.} for the containers. The \textsc{Service Resource} is based on the \textsc{Procedure},\sidenote{\eg{CBC, Basic Panel, HbA\textsubscript{1c}, \etc}} \ and \textsc{Collection Location} of the container.


\newthought{This gives Cerner} the ability to update the status of a sample automatically based on the last place it was \textsc{Logged-in.}

\begin{quote}
    \boldcap{For example: } Let's say a CBC and a HbA\textsubscript{1c} are sent to \textsc{SMCA Laboratory}.

    When they're \textsc{Logged-in} to \textsc{SMC Login}, the CBC have an \textsc{In-Lab} status,\sidenote{Meaning, it's available to be tested.} while the HbA\textsubscript{1c} will have \textsc{Pending} status.\sidenote{Meaning, it's been collected but hasn't been sent yet.}
\end{quote}


\newthought{In addition to tracking,} the \textsc{Service Resources} is used in \gls{pi} as the \textsc{Test Site}.

\section{How Does It Work?}

\subsection{Instruments and Benches and Sections oh my!}

Every test is \boldcap{Routed} to either an \treeitem{instrument}{Instrument} or \treeitem{bench}{Bench}.\\

\treeitem{instrument}{Instruments} are interfaced instruments.\sidenote{They can send the results to Cerner.}

\treeitem{bench}{Benches} are where manual tests are performed.

 Since there are so many instruments\sidenote{We should start a band!} and benches we need a way to organize them. To do this we have \treeitem{section}{Sections} and \treeitem{subsection}{Subsections}.\\

 \treeitem{section}{Sections} are departments in the lab.\sidenote{Heme, Chem, UA, \etc}

 \treeitem{subsection}{Subsections} are areas of the sections.\sidenote{Manual Chemistry, Tegs, Miscellaneous Micro, \etc}


\subsection{All That To Say:}

Tests are routed to a \treeitem{bench}{Bench} or \treeitem{instrument}{Instrument}.

\noindent
\treeitem{bench}{Benches} and \treeitem{instrument}{Instruments} belong to a \treeitem{subsection}{Subsection}.

\noindent
\treeitem{subsection}{Subsections} belongs to a \treeitem{section}{Section}.

% \section{Example of Routing}
% {\setstretch{0.5}\scriptsize\scshape
%     \cfinput{../routing/smcrouting.tex}
% }

% This chapter contains a breakdown of the Seton Network Laboratories and their routing.

% \section{Instruments and Benches}

% Every test is \boldcap{Routed} to either an \instrument or \bench.\\

% \instruments are interfaced instruments.\sidenote{They can send the results to Cerner.}

% \benches are where manual tests are performed.

% \section{Sections and Subsections}

%  Since there are so many instruments\sidenote{We should start a band!} and benches we need a way to organize them. To do this we have \sects and \subsects.\\

% \sects are departments in the lab.\sidenote{Heme, Chem, UA, \etc}

% \subsects are areas of the sections.\sidenote{Manual Chemistry, Tegs, Miscellaneous Micro, \etc}


% \subsection{All That To Say:}

% Tests are routed to a \bench or \instrument.

% \noindent
% \benches and \instruments belong to a \subsect.

% \noindent
% \subsects belongs to a \sect.
% \clearpage


{\setstretch{0.5}\scriptsize\scshape
    \tikzset{
      dirtree/.style={
        growth function=\dirtree@growth,
        every node/.style={anchor=north},
        every child node/.style={anchor=west},
        edge from parent path={(\tikzparentnode\tikzparentanchor) |- (\tikzchildnode\tikzchildanchor)},
      }
    }

    \IfFileExists{bhrouting.tex}{%
        \addcontentsline{toc}{chapter}{Routing Maps}
        \cfsection{Brackenridge}{./}{bhrouting.tex}
        \cfsection{Dell Children's}{./}{dcrouting.tex}
        \cfsection{Edgar B. Davis}{./}{sebdrouting.tex}
        \cfsection{Hays}{./}{shcrouting.tex}
        \cfsection{Highland Lakes}{./}{shlrouting.tex}
        \cfsection{Northwest}{./}{snwrouting.tex}
        \cfsection{Southwest}{./}{sswrouting.tex}
        \cfsection{Williamson}{./}{swcrouting.tex}
        \cfsection{Seton Medical Center}{./}{smcrouting.tex}
    }{%
        \label{sec:routing_maps}
        \cfsection{Brackenridge}{../routing}{bhrouting.tex}
        \cfsection{Dell Children's}{../routing}{dcrouting.tex}
        \cfsection{Edgar B. Davis}{../routing}{sebdrouting.tex}
        \cfsection{Hays}{../routing}{shcrouting.tex}
        \cfsection{Highland Lakes}{../routing}{shlrouting.tex}
        \cfsection{Northwest}{../routing}{snwrouting.tex}
        \cfsection{Southwest}{../routing}{sswrouting.tex}
        \cfsection{Williamson}{../routing}{swcrouting.tex}
        \cfsection{Seton Medical Center}{../routing}{smcrouting.tex}
    }
}}
}

\newcommand{\checkrouting}{\checkrefbookless{part:appendix}{\refpt{part:routing}}{\refpt{ch:appended_routing}}}

\newcommand{\dupnote}{\section{Duplicate Orders}

Cerner should be able to handle most instances when duplicate orders are placed. It won't allow the same \textsc{Orderable} to be placed twice.

If there are duplicate assays within the orderables, Cerner should automatically perform any cancellations or modifications as needed.


\importantblock{Duplicate checking occurs at several steps throughout the ordering process. \refpt{ch:doe_duplicate} for more information.}}




%%%%%%%%%%%%%%%%%%%%%%%%%%%%%%%%%
\let\Oldpart\part
\newcommand{\parttitle}{}
\renewcommand{\part}[1]{\Oldpart{#1}\renewcommand{\parttitle}{#1}}
%%%%%%%%%%%%%%%%%%%%%%%%%%%%%%%%%

\usepackage[toc, lot]{multitoc}
\usepackage[titles]{tocloft}
\renewcommand\cftpartfont{\LARGE\scshape\chaptercolor}
\renewcommand\cftchapfont{\large\scshape}
\renewcommand{\multicolumntoc}{2}
\setlength{\columnsep}{1cm}

\newcommand{\instrument}{\treeitem{instrument}{Instrument}}
\newcommand{\instruments}{\treeitem{instrument}{Instruments}}
\newcommand{\bench}{\treeitem{bench}{Bench}}
\newcommand{\benches}{\treeitem{bench}{Benches}}
\newcommand{\sect}{\treeitem{section}{Section}}
\newcommand{\sects}{\treeitem{section}{Sections}}
\newcommand{\subsect}{\treeitem{subsection}{Subsection}}
\newcommand{\subsects}{\treeitem{subsection}{Subsections}}
\newcommand{\site}{\treeitem{site}{Site}}



% {\parbox[c]{.55\textwidth}{}\parbox[c]{.4\textwidth}{\includegraphics{{\gfx #2}}}}


% \newcommand{\miniappicon}[1]{\includegraphics[width=1em]{\gfx #1}}

\newcommand{\miniappicon}[1]{\tcbox[%
       enhanced,
       arc=3pt,
       borderline={2pt}{0pt}{deeppurple500},
       size=tight,
       nobeforeafter]{%
 \includegraphics[height=1em]{\gfx #1}} }










%%%%%%%%%%%%%%%%%%%%%%%%%%%%%%%%%%%%%%%%%%%%%%%%%%%%%%%%%%%%%%%%%%%%%%%%
% Activity Functions
%%%%%%%%%%%%%%%%%%%%%%%%%%%%%%%%%%%%%%%%%%%%%%%%%%%%%%%%%%%%%%%%%%%%%%%%



\newcommand{\sideref}[2]{
    \sidenote{\activityRef{#1}{#2}}
}

\newcommand{\refBlock}[2]{
  \infoblock{For help with the next activity; see:

  \activityRef{#1}{#2}}

}

\newcommand{\turnto}[2]{
  \infoblock{Have them turn to:

  \activityRef{#1}{#2}}
}

\newcommand{\marginref}[2]{
    \marginnote{\activityRef{#1}{#2}}
}


\smartinput{activity_functions.tex}