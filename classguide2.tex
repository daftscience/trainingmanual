\setcounter{page}{1}
\setitemize{noitemsep,topsep=0pt,parsep=0pt,partopsep=0pt, label={-}}

\chapter{Quick reference}

In order to find your information quicker, you can use this page to answer some of the questions.


\bigskip
\bigskip
\answer{Your test patient's MRN\label{your_mrn}}

\bigskip
\answer{Accession Number (Blood)\label{blood_accession}}
\answer{Accession Number (Bone Marrow)}
\answer{Accession Number (Synovial)\label{synovial}}

\bigskip
\answer{Which container is the PTT on?\label{ptt_container}}
\answer{Which \boldcap{Service Resource} is the \boldcap{PTT} routed to?\label{ptt_sr}}

\bigskip
\answer{What is the number of the list your PTT is on?\label{ptt_list}}
\answer{What is the number of the list your CHABM is on?\label{bm_list}}

\clearpage


    \chapter{Getting Started}
    \section{Opening the App-Bar}
    \begin{itemize}
        \item Open Presentation to \boldcap{Log In Instructions}
        \item Pass Around Sign up Sheets.
        \item Make Sure Everyone can Log-in.
        \issues{
            \item Not using \boldcap{User name for Password}
            \item Multiple App-Bars open
        }
        \item Assign MRNS
            \telltowrite{The MRN's}
    \end{itemize}


    \section{About the Class}
    \begin{itemize}
        \item \boldcap{Objectives}
            \begin{itemize}
                \item Course will cover most applications
                \item Some Exceptions: Registration Order Entry, Collections Inquiry
            \end{itemize}
        \item \boldcap{Warnings}
            \begin{itemize}
                \item Leaving this class they will \boldcap{not} know Cerner.
                \item They need to practice
                \item Make sure they can get off bench time
                \item Use the Manual
                \item We can provide them with help
            \end{itemize}
    \end{itemize}
    \section{Accession Numbers}
        \begin{itemize}
            \item Represents a single collection.
                \begin{itemize}
                    \item One number is given to all \boldcap{Like} specimen.\sidenote{Lav, Gold, Blue, Grn etc.}
                    \item \boldcap{Unlike} specimen will be on separate accessions.\sidenote{Body Fluids, Urines, etc.}
                    \item Microbiology samples will be on separate accession numbers.
                \end{itemize}
            \item Accessions are unique.\sidenote{Unlike Sunquest, they don't recycle.}
            \item Based on the Julian Date
        \end{itemize}
        \begin{itemize}
            \item Each Container has a Container ID.
            \item Instruments will only run the tests assigned to a container.
            \item It's important not to swap labels.
        \end{itemize}
    \subsection{Downtime Accessions}
    \wip
        \begin{itemize}
            \item Julian Day's start at 400
            \item No container ID
            \item They basically work like Z Labels.
        \end{itemize}

\section{Routing}
    % \subsection{Service Resources}
        \begin{itemize}
            \item Routing allows Cerner to Track Samples.
            \item It works using \boldcap{Service Resources}\sidenote{The place where samples are performed.}
        \end{itemize}
        \turnto{part:routing}{sec:routing_maps}\marginnote{This reference is weird. Basically, it's the routing maps.}
        \begin{itemize}
            \item \boldcap{Service Resources} are locations within the lab where a tests are performed.\sidenote{Instrument or Bench.}
            \item Each test is assigned (Routed to) a \boldcap{Service Resource}\sidenote{Usually, an instrument or bench.}
            \item If it can't be performed in the hospitals lab it will be routed to another site.
        \end{itemize}
    \subsection{Hay's Example}
        \begin{itemize}
            \item \textit{Walk through Hays Example containers and their routing.}\sidenote{SMC is an going out of the lab.}
            \item Each \boldcap{Service Resource} is assigned an \boldcap{In-Lab Location}
            \item \textit{Show the routing}
        \end{itemize}

    \subsection{In-Lab Location}
    \begin{itemize}
        \item Each \textsc{Service Resource} has an \boldcap{In-Lab Location.}
        \item This is where the sample needs to be \boldcap{Logged-in}\sidenote{Remember ``In-Lab'' means ``Received.'' This is where the sample needs to be received. }
    \end{itemize}
    \subsection{Analogy}
        \begin{itemize}
            \item Tube is addressed to an instrument
            \item Mail is addressed to a person
            \item Mail needs to be delivered to house, not person.
            \item Sort of like tubs needs to be logged into the laboratory.
            \item \boldcap{In-Lab is the most important thing to know}
        \end{itemize}

    \subsection{Workflow Example 1}
        This should be pretty self explanatory.

        \infoblock{Notice the SMC Sample is routed to \boldcap{SMC}}

    \subsection{Workflow Example 2}
        Very Similar to the previous example.

        \infoblock{If the In-lab Location is out of network: Basically, all they need to know is that we send it to \boldcap{referrals}.}

\switchto{Cerner}

\chapter{Department Order Entry}
    \section{Set Defaults}
    \begin{itemize}
        \item Tell them this is in the book for future reference.
        \item \boldcap{Task}\faArrowRight\boldcap{Define Default Values}
            \begin{itemize}
                \item Select \boldcap{Patient Identifier} Field.
                \item \boldcap{Never} Set \textsc{Client}.\sidenote{It limits searching.}
                \item Label Printers. \boldcap{LEAVE BLANK FOR TODAY} but for go-live
                    \begin{itemize}
                        \item \boldcap{Label Printer} Where your labels will print. Set this
                        \item \boldcap{Aliquot Printer} Set to the same as above.
                        \item \boldcap{Media Label Printer} Where micro labels will print.
                    \end{itemize}
            \end{itemize}
    \end{itemize}

\section{Placing an Order}
    \subsection{Search for a Patient}
        \begin{itemize}
            \item \boldcap{Right Click} the \textsc{Person Name} field.
            \item Select \boldcap{MRN}\sidenote{Alternatively, \boldcap{Ctrl+Tab} to cycle through them.}
            \item Enter their test patient's MRN.
            \item Hit \boldcap{Enter}
            \infoblock{At this point, the demographics should populate with the patient's information.}
        \end{itemize}

    \importantblock{Instruct everyone to stop using Cerner, and watch what you're doing.}

    \subsection{Finding Orderables}
        \begin{itemize}
            \item Click the \boldcap{Orderable} field.
            \item Search for ``Lytes''
            \warnblock{Have them make sure to check that the orderable matches the thing they're looking for. \eg{Teg will NOT populate with Thromboelastography.}}
            \item Explain Ancillary verses Primary orderables.
        \end{itemize}

    \subsection{Selecting Options}
        \begin{itemize}
            \item Yellow Fields are Required.
            \item Select \boldcap{Collected} if it's not already checked.
            \item Make sure \boldcap{Receive Location} is set to SMC Login. \sidenote{Go live it will be their hospitals location.}
        \end{itemize}

\importantblock{Leave the window up and allow everyone to get to where you are.}

    \subsection{Scratch-Pad}
        \begin{itemize}
            \item Add the \boldcap{Lytes} to the scratch-Pad.\sidenote{Ctrl+A}
            \item Add the rest
            \begin{itemize}
                \item \boldcap{hCG Quant}
                \item \boldcap{TNSIBC}
                \item \boldcap{CHABM}
            \end{itemize}
            \item Submit the orders.\sidenote{Ctrl+O}
            \telltowrite{Both Accession Numbers}
            \item Order a CDSF
            \infoblock{If running behind, just demonstrate the error.}
            \item Change the received time to be after the collection time.
            \item Click the little X icon.
        \end{itemize}



    \section{Accession Add-On}
        \begin{itemize}
            \item Add-on a PTT
        \end{itemize}

\importantblock{Keep the \boldcap{Override} option open, and wait for everyone to catch up.}

    \subsection{To Override or Not to Override}
        Think of this question as:
        \begin{quote}
            Can I add this test onto the existing Containers?

            We have a \boldcap{Green} and a \boldcap{GOld}. Can I add a PTT to a green or a gold?
        \end{quote}
        \begin{itemize}
            \item Click Yes and order it.
        \end{itemize}


\chapter{Container Inquiry}

\begin{itemize}
    \item Have them pull up their Blood Accession Number
\infoblock{They will be missing their \boldcap{Container List} show them how to fix it.}

\end{itemize}

\switchto{Presentation}

\begin{itemize}
    \item When to use it
    \item Events
\end{itemize}

\switchto{Cerner}

\begin{itemize}
    \item Explain Containers
        \telltowrite{PTT Container ID}
    \item Explain Events
        \infoblock{Have them make sure all their containers have a location of \boldcap{SMC Login}}
    \item Show Routing
        \telltowrite{PTT Service Resource.}
\end{itemize}

\section{Troubleshooting}
\begin{itemize}
    \item When they run into issues:
    \begin{itemize}
        \item Check the last event \boldcap{Location}
        \item Check that the \boldcap{In-Lab Location} matches.
    \end{itemize}
\end{itemize}





\switchto{Presentation}

\chapter{Accession Result Entry}

\begin{itemize}
    \item Modes
    \item Result Flags
        \begin{itemize}
            \item Review Flag = Mostly for auto-verification.\sidenote{If it \boldcap{Doesn't} have a review flag, it \boldcap{Still} needs to be reviewed.}
            \item Linearity - Outside of our reportable range. This is \boldcap{After} dilution.\sidenote{Remisol handles dilutions}
            \item Notify - This is used for E.D. troponins that are positive. Holds auto-verification.
        \end{itemize}
\end{itemize}

\switchto{Cerner}

\section{Default Values}
    \infoblock{It helps if you leave these menus open for a bit so they can see.}
\begin{itemize}
    \item View \faArrowRight Customize Result Display:
    \begin{itemize}
        \item Select Reference Range
        \item Select Service Resource
    \end{itemize}
    \item View \faArrowRight Customize \faArrowRight Sort Assays \faArrowRight By Service Resource.
\end{itemize}

\section{Entering Results}

\begin{itemize}
    \item Enter the following results for the Lytes:\sidenote{Don't Verify}
    \begin{itemize}
        \item Sodium: 201
        \item Potassium: 6
        \item Chloride: 100
        \item CO2: 20
    \end{itemize}
\end{itemize}

\importantblock{Instruct everyone to stop using Cerner, and watch what you're doing.}

\begin{itemize}
    \item Notice Anion gap is  gt 80.
    \item Show what happens if CO2 is 0\sidenote{Multiple Inequalities.}
    \item Most calculations will take place in Cerner.
    \item Right Click Options:
    \begin{itemize}
        \item We can view Calculations\sidenote{Right Click, View Equation}
        \item Show Procedure Information
        \item Show Convert Result
        \item Show all previous
        \item Show Comments
        \begin{itemize}
            \item Comment Vs Note
            \item Result Comment vs Order Comments\sidenote{Order means Orderable not the whole accession number.}
            \item Result Comments are considered part of the result, they cannot be added after the result has been verified.
        \end{itemize}
    \end{itemize}
\end{itemize}

\importantblock{Everyone can Follow along now.}

\begin{itemize}
    \item Quickly note that you can branch with this application.
    \item Click Verify
    \item Click Yes or Ok
    \item If they click ``No'' it won't verify or perform.
    \begin{itemize}
        \item Walk them through the Comments\sidenote{All the templates are on page 239}
    \end{itemize}
    \item Cerner Will not allow you to verify or perform a critical result without a result comment or note.
\end{itemize}

\subsection{Correcting Results}

\begin{itemize}
    \item Open Accession Number using Correction mode.
    \item Notice you can't modify things without results.
    \item Notice the little Icon.
    \item Change the \boldcap{Sodium} to 136.
    \item Noticed the AGAP Changed.
    \item Enter Batch Comment \boldcap{All Selected}
    \item If you fail to enter a Correction Comment it will need to be corrected again.
\end{itemize}

\chapter{Order Result Viewer}

\section{Set Defaults}
    \begin{itemize}
        \item Set Look Back to 30 days
        \item Uncheck All Dates
    \end{itemize}
    \begin{itemize}
        \item View \faArrowRight Customize \faArrowRight Orderlist
        \item Add the three Cancel Options.
        \item This takes a while....
    \end{itemize}

\section{Pull up Order}

\begin{itemize}
    \item Click The Blue Arrow
    \item Search for MRN
    \item Status is helpful for troubleshooting
        \begin{itemize}
            \item Page 106 has all the order statuses
            \item Scheduled
            \item Dispatched
            \item Collected
            \item In Transit
            \item In Lab
            \item In Process
            \item Completed
            \item Canceled.
        \end{itemize}
    \item Double click to view Results
    \begin{itemize}
        \item Comments
        \item Previous Results
        \item Result History
    \end{itemize}
\end{itemize}

\section{Cancel Order}
\begin{itemize}
    \item Select SHCG
    \item Note Reason (Required)
    \item Add Comment
\end{itemize}

\chapter{Pending Inquiry}
    \section{Pulling a pending}
        \begin{itemize}
            \item About the Search Window
            \begin{itemize}
                \item Test Site is any service resource.\sidenote{Usually they will want to pull a section or subsection.}
                \item Type:
                    \begin{itemize}
                        \item All Pending will show things which have not gotten to the lab
                        \item Received only - Best not to use this
                        \item In-Lab Only: It shows everything you need to result.
                        \item Scheduled shows things that don't have accessions yet.
                    \end{itemize}
            \end{itemize}
            \item Pull a pending for SMC Coag
            \item this doesn't refresh, click arrow to do that.

        \end{itemize}




\chapter{Transfer Specimens}

\chapter{Specimen Login}

\chapter{Modify Collections}


\switchto{Presentation}
\importantblock{CLAs can Leave}

\chapter{}










\part{Class Activities}
\section{Department Order Entry}

\warnblock{Please read the instructions carefully!
    \begin{itemize}
        \item Do \boldcap{NOT} change the client field.
        \item Do \boldcap{NOT} submit orders until instructed to do so.
    \end{itemize}
}

\refBlock{part:doe}{patientLookup}
\activity{Search for your Patient using the MRN}\sidenote{\refQuestion{your_mrn}}

\activity{Search for a \boldcap{Electrolytes Metabolic Panel}}

\infoblock{Set the following values:
    \begin{itemize}
        \item{{\faCheckSquareO} Collected}
        \item Set the \boldcap{Received Location} to SMC Login.
    \end{itemize}

}

\activity{Add the Electrolytes Panel to your \gls{spad}}\sidenote{\hotkey{\boldcap{Ctrl+A}}}

\infoblock{For the next activity, you don't need to change any of the options.They will default to those chosen with the Electrolytes Panel.}

\activity{Add the following tests:}


\begin{quote}
\begin{quote}
    \begin{itemize}
        \item hCG Qualitative
        \item TNSIBC
        \item Chromosome Analysis on Bone Marrow
    \end{itemize}
\end{quote}
\end{quote}

\activity{Submit the Order}\sidenote{\hotkey{\boldcap{Ctrl+O}}}

\answerFP{
\item Accession Number (Blood)
\item Accession Number (Bone Marrow)}

\subsection{Creating An Error}


\importantblock{For this next activity, set the \boldcap{Collection Time} to be greater than the \boldcap{Received time}.}

\activity{Order a CDSF}\sidenote{This will generate an ERROR.}

\refBlock{part:doe}{sec:submission_errors}

\activity{View the Submission Error}

\activity{Fix the error, and re-submit the orders}

\answerFP{
    \item Accession Number (Synovial)
}


\subsection{Accession Add-On}

\activity{Add a PTT to the blood sample ordered earlier.}

\section{Container Inquiry}

\activity{Using \boldcap{Container Inquiry} search for the blood sample.}\sidenote{\refQuestion{blood_accession}}

\answerFP{\item What container is your PTT on?}

\answerFP{\item Which \boldcap{Service Resource} is the PTT routed to?}
\answerkey{SMC ACL1}



\chapter{Entering Results}

\section{Accession Result Entry}

\subsection{Getting Started}

\refBlock{part:are}{ch:are_setup}

\activity{Set the Default Values}


\subsection{Entering Results}

\importantblock{Do not \boldcap or \boldcap{Verify} these until instructed.}

\activity{Open the Accession number for your Blood Sample using ARE.}\sidenote{\refQuestion{your_mrn}}

\activity{Set the \boldcap{Test Site} to SMC Remisol.}\sidenote{Click Retrieve if needed.}

\activity{Enter results for the Lytes using the following criteria:}

\begin{quote}
\begin{quote}
    \begin{itemize}
        \item \boldcap{Sodium:} 201
        \item \boldcap{Potassium:} 6
        \item \boldcap{Chloride:} 100
        \item \boldcap{CO2:} 20
    \end{itemize}
\end{quote}
\end{quote}


\activity{Verify the panel.}

\importantblock{Normally, we select ``Yes.''  However, in order to show you additional features, we'd like you select ``No.''}

\activity{Click ``No''}

\activity{Enter a \boldcap{Batch Comment}}\sidenote{\hotkey{\boldcap{Ctrl+B}}}

\activity{Using the \boldcap{CallRed} template, enter a critical comment}

\refBlock{part:comments}{ch:using_templates}

\activity{Click Verify}

\subsection{Correction Mode}

In this section, we will be modifying some results.\\

\activity{Set ARE to \boldcap{Correction Mode}}

\activity{Change the Sodium to 136}

\activity{Enter a Correction Comment}\sidenote{\hotkey{\boldcap{Ctrl+A}}}

\chapter{Viewing Orders}


\section{Order Result Viewer}

\refBlock{part:orv}{ch:orv_getstart}

\activity{Change default values}

\subsection{Searching for a Patient}

\activity{Pull up your patient's orders}\sidenote{\refQuestion{your_mrn}}

\activity{Double click on the Electrolytes Panel.}

\activity{Select a \boldcap{Corrected Result}}\sidenote{Look at the \boldcap{Status} column.}

\activity{Click on \boldcap{History}}

\subsection{Canceling Orders}

\refBlock{part:orv}{ch:orv_cancel}

\activity{Cancel the SHCG}

\section{Pending Inquiry}

\refBlock{part:pi}{ch:customizing_pending}

\activity{Modify the headings}

\activity{Re-Route the PTT to Brakenridge.}

\refBlock{part:pi}{ch:transfer_from_pending}

\activity{Transfer the PTT.}


\chapter{Transfer Specimen}

\activity{Set up defaults}

\infoblock{In this section we will create two transfer lists. Please pay close attention, this section gets complicated.}


\section{Creating Lists}

\subsection{Your First List}

The first list will contain your PTT.

\importantblock{Use the Following Settings:

From: \boldcap{\Large From SMC Login}

To: \boldcap{\Large To BH Login}

}

\refBlock{part:transfer}{ch:creat_trans_list}

\activity{Create a transfer list for the PTT Order.}

\answerFP{\item What is your list number?}

\subsection{Your Second List}

The second list will contain your CHABM.

\importantblock{Use the Following Settings:

From: \boldcap{\Large From SMC Login}

To: \boldcap{\Large To BH Referrals}

}

\refBlock{part:transfer}{ch:creat_trans_list}

\activity{Create a transfer list for the CHABM}

\answerFP{\item What is your list number?}

\subsection{Modify List}

\activity{Select and Modify your First List}\sidenote{\refQuestion{ptt_list}}

\activity{Add the TNSIBC To the list}\sidenote{\refQuestion{blood_accession}}

\chapter{Specimen Login}

\section{Log-in By Accession}

\refBlock{part:login}{sec:login_procedure}

\activity{Open Accession Log-in and choose: \boldcap{Log-in by Accession} }

\activity{Make Sure the \textsc{Log-in Location} is set to \boldcap{BH Login}}

\importantblock{In order to prevent errors, enter the container ID after the Accession Numbers.\sidenote{Capital Letters}}

\begin{itemize}
    \item CHABM\sidenote{This will be container `A'}
    \item PTT\sidenote{\refQuestion{ptt_container}}
\end{itemize}

\section{Login By List}

\refBlock{part:login}{ch:login_list}

\activity{Open Accession Log-in and choose: \boldcap{Log-in by List}}

\activity{Log in your Second List}\sidenote{\refQuestion{bm_list}}

\activity{Log in your First List}

\chapter{Advanced ARE}

\activity{Open your Synovial Fluid in ARE}

\activity{Result it so that a Differential and Crystal ID will reflex.}

\infoblock{\begin{itemize}
        \item \boldcap{WBC} > 5
        \item \boldcap{Crystals}: Present
    \end{itemize}
}

\section{Differential Mode}

\refBlock{part:are}{sec:perform_differential}
\activity{Perform a Differential.}

\activity{Switch to Accession Mode and enter the Total Cells Counted.}

