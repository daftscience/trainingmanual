There are a few applications within Cerner\sidenote{\gls{are} and \gls{doe} are the two most common applications with this feature.} which display demographic information.

The information displayed can be modified to display any necessary information. This can be very useful if you find yourself searching for specific patient information.\\

\prettyimage{width=\textwidth}{graphics/demographics}


\section{Helpful Tips}
Before you begin, here are a few tips you should keep in mind if you decide to create your own demographics setup.

\newthought{Follow these tips} for the best results.
\begin{itemize}
    \bolditem{Number of Columns} The fewer the columns the better.\sidenote{Fewer columns means information will not be cutoff.}
    \bolditem{Avoid TMI} Only include fields you \textit{know} you'll need.
    \bolditem{Split Locations} Separate the patients floor from the room number.
\end{itemize}


\section{Changing Demographics}

\paragraph{Click} \boldcap{View} from the menu bar.\\

\prettyimage{width=.5\linewidth}{graphics/view_menu2}

\paragraph{Click} \boldcap{Customize\ldots}

\paragraph{Click} \boldcap{Demographics\ldots}\\

\prettyimage{width=.5\textwidth}{graphics/demographics_menu}

\newthought{This window} displays the \textsc{Values} currently set in each of the demographics \textsc{Fields}.\\

\prettyimage{width=\textwidth}{graphics/modify_demo}

\paragraph{Set} the columns count to \boldcap{2}.\sidenote{It's located on the top left corner of the window.}\\


\prettyimage{width=.25\textwidth}{graphics/demo_column_count}

\newthought{Cerner will ask} if you're sure.\\

\prettyimage{width=.5\textwidth}{graphics/change_demo_warning}

\paragraph{Click} \btn{graphics/yes}\\

\newthought{Next to each field} in the \textsc{Demographics Data Elements} box is an ellipses \btn{graphics/ellipses} button. Clicking it will open the \textsc{Set Demographic Data Elements window.}

The two important fields are \textsc{Value}\sidenote{\textsc{Value} is what information will be displayed.} and \textsc{Label}.\sidenote{\textsc{Label} is the name it is given. This can be customized.}\\


\noindent
\begin{tikzpicture}
\node [anchor=west] (val) at (-1,4.0) {\scshape{Value}};
\node [anchor=west] (lb) at (-1,2.0) {\scshape{Label}};
\begin{scope}[xshift=1.5cm]
    \node[anchor=south west,inner sep=0] (image) at (0,0) {\prettyimage{width=.7\textwidth}{graphics/set_demo_data}};
    \begin{scope}[x={(image.south east)},y={(image.north west)}]
        \draw [-stealth, line width=3pt, deeporange500] (val) to[out=0, in=-180] (0.02,0.82);
        \draw [-stealth, line width=3pt, teal500] (lb) to[out=0, in=-180] (0.02,0.33);
    \end{scope}
\end{scope}
\end{tikzpicture}%

\newthought{For each of the 10} fields listed:
\begin{description}
    \paraitem{Click} the \btn{graphics/ellipses} button.
    \paraitem{Set} the \textsc{Values} to match the image below.
    \paraitem{Modify} the \textsc{Label}, if needed.\sidenote{Sometimes it's helpful to make it shorter.}
\end{description}

\prettyimage{width=\textwidth}{graphics/best_demographics_settings}

\paragraph{Click} \btn{graphics/ok} when finished.


\section{Restore to Defaults}

At any the values can be restored to their original settings.

\begin{description}
    \paraitem{Click} the \btn{graphics/restore} button.
\end{description}





